\documentclass{beamer}

\usetheme{Copenhagen}

\title{On the minimal number of generators of a finite group}
\author{Diogo Santos}
%\date{september 2024}

\begin{document}
\maketitle

\begin{frame}
    \frametitle{Introduction}
    \begin{itemize}
        \item<1-> Finding the minimal number of gentators of a finite group $H$ \\
    \end{itemize}
    \onslide<2->{Can be reduced to:}
    \begin{itemize}
        \item<2-> Finding the minimal number of gentators of a finite group $H$ such that $d(H/N) \le m$ for every non-trivial normal subgroup $N$, but $d(H) > m$\\
    \end{itemize}

\end{frame}

\begin{frame}{The case $m = 1$}
    \begin{theorem}
        Let $H$ be a finite nilpotent group such that $d(H/N) \le 1$ for every non-trivial normal subgroup $N$, but $d(H) > 1$. Then $H \cong \mathbb{Z}_p \times \mathbb{Z}_p$ for some prime $p$.
    \end{theorem}

    \begin{proof}<2->
        \renewcommand{\qedsymbol}{} % Remove the QED symbol for this proof
        \begin{itemize}
            \item $H = P_1 \times \ldots  \times P_n$ where $P_i$ is a Sylow $p_i$-subgroup for $1 \le i \le n$ and $p_1,\ldots ,p_n$ are distinct primes.
            \item<3-> If $P_1,\ldots , P_r$ are cyclic, we obtain $H \cong \mathbb{Z}_{p_1\ldots p_n}$  which contradicts $d(H) > 1$.Without loss of generality we can thus assume that $P_1$ is not cyclic.
            \item<4-> $n \ge 2 \implies P_1 \cong H/(1 \times P_2 \ldots  \times P_n)$ and thus $d(P_1) = d(H/(1 \times P_2 \ldots  \times P_n)) = 1$, contradiction.
            \item<5-> By Theorem \ref{th:fratgen}, $\Phi(H) = 1$ hence $H = (\mathbb{Z}_{p_1})^q$ by Theorem \ref{fratpgroup}.
        \end{itemize}
    \end{proof}
\end{frame}

\begin{frame}{The case $m = 1$}
    \begin{proof}
        \begin{itemize}
            \item<1-> $q = 2$ since $$q-1 = d((\mathbb{Z}_{p_1})^{q-1}) = d(H/(\mathbb{Z}_{p_1} \times 1 \times \ldots  \times 1)) = 1.$$
        \end{itemize}

    \end{proof}

\end{frame}

\end{document}