\documentclass{beamer}

\usetheme{Copenhagen}

\title{On the minimal number of generators of a finite group}
\author{Diogo Santos}
%\date{september 2024}

\newcommand{\subgen}[1]{\langle #1 \rangle}
\newcommand{\card}[1]{| #1 |}
\newcommand{\nsub}{\triangleleft}
\newcommand{\cmpl}[1]{\overline{ #1 }}
\newcommand{\dt}[2]{(#1)_{#2}}
\newcommand{\soc}[1]{soc {(#1)}}
\newcommand{\diag}[1]{diag({#1})}
\newcommand{\Mi}[1][i]{M_{#1}}
\newcommand{\dl}[0]{\dot{l}}
\newcommand{\core}[2][]{core_{#1}({#2})}
\newcommand{\aut}[1]{Aut\, {#1}}
\newcommand{\inn}[1]{Inn\, {#1}}

\begin{document}
\maketitle

\begin{frame}
    \frametitle{Introduction}
    \begin{itemize}
        \item<1-> Finding the minimal number of gentators of a finite group $H$ \\
    \end{itemize}
    \onslide<2->{Can be reduced to:}
    \begin{itemize}
        \item<2-> Finding the minimal number of gentators of a finite group $H$ such that $d(H/N) \le m$ for every non-trivial normal subgroup $N$, but $d(H) > m$\\
    \end{itemize}

\end{frame}

\begin{frame}{The case $m = 1$}
    \begin{theorem}
        Let $H$ be a finite nilpotent group such that $d(H/N) \le 1$ for every non-trivial normal subgroup $N$, but $d(H) > 1$. Then $H \cong \mathbb{Z}_p \times \mathbb{Z}_p$ for some prime $p$.
    \end{theorem}

    \begin{proof}<2->
        \renewcommand{\qedsymbol}{} % Remove the QED symbol for this proof
        \begin{itemize}
            \item $H = P_1 \times \ldots  \times P_n$ where $P_i$ is a Sylow $p_i$-subgroup for $1 \le i \le n$ and $p_1,\ldots ,p_n$ are distinct primes.
            \item<3-> If $P_1,\ldots , P_n$ are cyclic, we obtain $H \cong \mathbb{Z}_{p_1\ldots p_n}$  which contradicts $d(H) > 1$.Without loss of generality we can thus assume that $P_1$ is not cyclic.
            \item<4-> $n \ge 2 \implies P_1 \cong H/(1 \times P_2 \ldots  \times P_n)$ and thus $d(P_1) = d(H/(1 \times P_2 \ldots  \times P_n)) = 1$, contradiction.
            \item<5-> Since $d(H) = d(H/\Phi(H))$, $\Phi(H) = 1$
        \end{itemize}
    \end{proof}
\end{frame}

\begin{frame}{The case $m = 1$}
    \begin{proof}
        \begin{itemize}
            \item<1-> $H \cong H/\Phi(H)$ is a $\mathbb{Z}_{p_1}$-vector space and thus $H = (\mathbb{Z}_{p_1})^q$
            \item<2-> $q = 2$ since $$q-1 = d((\mathbb{Z}_{p_1})^{q-1}) = d(H/(\mathbb{Z}_{p_1} \times 1 \times \ldots  \times 1)) = 1.$$
            
        \end{itemize}

    \end{proof}

\end{frame}

\begin{frame}{The groups $L_k$}
    Throughout $L$ will always denote a finite group with a unique minimal normal subgroup $M$. Furthermore if $M$ is abelian, we also assume that $M$ is complemented in $L$.
    \onslide<2->{\begin{definition}
        Given a positive integer $k$, the group $L_k$ is a subgroup of $L^{k}$ defined by:
        $$
        L_k = \{ (l_1,\ldots ,l_k) \in L^k | l_1M = \ldots  = l_kM \}.
        $$
    \end{definition}}
    \vspace{1cm}
    \onslide<2->{The group $L_k$ can be described as $\diag{L^{k}}M^{k}$.}

\end{frame}

\begin{frame}{Properties of $L_k$}
    \begin{itemize}
        \item<2-> $\soc{L_k} = M^k$
        \item<3-> $L_k / M^k \cong L/M$
        \item<4-> If $M$ is abelian and complemented by $C$ in $L$, then $M^k$ is complemented by $\diag{C^k}$
        \item<5-> The quotient of $L_{k+1}$ by any of its minimal normal subgroups is isomorphic to $L_k$
        \item<6-> The sequence $d(L_k)_{k \in \mathbb{N}}$ is unlimited and non-decreasing.
        \item<7-> For all $k \in \mathbb{N}$, $L_{k+1} \le L_{k}$.
    \end{itemize}
    
\end{frame}

\begin{frame}{The function $f$}
    \begin{definition}
        Given a group $L$ we define $f(L, m) = k+1$ if and only if $d(L_k) = m < d(L_{k+1})$. When $L$ can be identified from the context, we denote $f(L,m)$ as $f(m)$.
    \end{definition}
    \begin{itemize}
        \item<2-> Thus the function $f$ gives us \textit{the integer $k+1$ for which any proper quotient of $L_{k+1}$ has minimal number of generators smaller or equal to $m$ but $d(L_{k+1}) > m$.}
        \item<3-> It follows from the last $2$ properties of the groups $L_k$ that this function is non-decreasing and unbounded. 
    \end{itemize}
\end{frame}

\begin{frame}{General Case}
    \begin{theorem}
        Let $m$ be an integer with $m \ge 1$ and $H$ a finite group such that $d(H/N) \le m$ for every non-trivial normal subgroup $N$, but $d(H) > m$. Then there exists a group $L$ which has a unique minimal normal subgroup $M$ and is such that $M$ is either non-abelian or complemented in $L$ and $H \cong L_{f(L,m)}$.
    \end{theorem}    

\end{frame}

\begin{frame}{The function $f$}
    
\end{frame}
    \begin{theorem}
        Let $m \ge d(L)$ and $q$ be the number of $(L/M)$-endomorphisms of $M$ when $M$ is abelian. Then 
        $$
        f(m) = 1 +
        \begin{cases}
            \phi_L(m)/(\card{\Gamma}\phi_{L/M}(m))  \text{ if } M \text{ is not abelian,} \\
            log_q(1+(q-1)\phi_L(m)/\card{\Gamma}\phi_{L/M}(m))  \text{ if } M \text{ is abelian.} \\
        \end{cases}
        $$
    \end{theorem}
\end{document}